% NSF proposal generation template style file.
% based on latex stylefiles written by Stefan Llewellyn Smith and
% Sarah Gille, with contributions from other collaborators.
%
\documentclass{article}

% See this file for a set of pre-defined journal abbreviations



% This handles hanging indents for publications


\begin{document}

\begin{center}
{\Large{\bf Heath RCT and Experiment Literature Review}\\*[3mm]
{\bf Atheylab -Dr.Consulta Project} \\*[3mm]

Susan Athey\\
Niall Keleher 

\end{center}


For the following articles below the country, research question, type of intervention, sample size, estimated effect size with standard errors(SE) and if any heterogeneous treatment effect are mentioned(HTE).

\hfill


\textbf{Innovations for Poverty Action Papers}

\textit{Precommitment, Cash Transfers, and Timely Arrival for Birth: Evidence from a Randomized Controlled Trial in Nairobi Kenya}

\hfill

Country: Nairobi, Kenya

\hfill

Research Question:
Whether cash transfers, enhanced with behavioral “nudges,” can help women plan there delivers better and arrive at facilities earlier in deliver.

\hfill

Type of Intervention: precommitment transfer package which bundles a labeled cash transfer and  precommitment conditional transfer - cash and label that attempts to nudge women to plan for pregnancy better. 
Women in treatment given money up front to help lower cost of traveling to desired facility and make a commitment to deliver there later. (LCT)
Then treatment women could earn an extra payment if they did deliver at the place they stated. (L-CCT)

\hfill

Sample Size:336

\hfill

Estimated Effect \& Standard Errors(SE): Measures was time between contractions based on index created for study.
Treatment: 2.99 (1.32) more minutes between contraction at departure for hospital 

2.06 (1.32) more minutes between contraction at arrival for hospital 

Dilation at first exam was -.27cm (.33) less for treatment

Time in hospital before birth was longer on average for treatment 1.07 minutes (1.11)

\hfill

Any HTE: No

\hfill

\break

\textit{Measuring The Impact Of Cash Transfers And Behavioral ‘Nudges’ On Maternity Care In Nairobi, Kenya}
(sister study to above)

\hfill

Country: Nairobi, Kenya

\hfill

Research Question:
Whether cash transfers, enhanced with behavioral “nudges,” can help women deliver in better quality facilites.

\hfill

Type of Intervention: precommitment transfer package which bundles a labeled cash transfer and  precommitment conditional transfer - cash and label that attempts to nudge women to plan for pregnancy better. 
Women in treatment given money up front to help lower cost of traveling to desired facility and make a commitment to deliver there later. (LCT)
Then treatment women could earn an extra payment if they did deliver at the place they stated. (L-CCT)

\hfill

Sample Size:418

\hfill

Estimated Effect \& Standard Errors(SE): 

\hfill

No Standard Errors reported
Patient-reported facility quality of care 

Good communication skills of health care workers 68.7 LCT 6.2 L-CCT 14.1*** 
 
Never disrespected or abused at the facility 81.7 LCT 7.1* L-CCT 7.3*

Nontechnical quality index (average of all components) 78.8 5.5 L-CCT 7.0**
 
Patients delivering at a facility that meets better for: N = 363

Routine newborn care LCT 14.5**  L-CCT 10.0*

Basic emergency newborn care L-CCT 14.6*** 

Comprehensive emergency newborn care L-CCT 15.1*** 

\hfill

Any HTE: No


\hfill

\textit{Social Signaling and Childhood Immunization: A Field Experiment in Sierra Leone}


\hfill

Country: Sierra Leone

\hfill

Research Question:
 Can we increase timely and complete vaccination, by allowing parents to signal to others that they vaccinated their child?


\hfill

Type of Intervention: Handout different colored bracelets at different stages of vaccination for children in-order to create an explicitly visible signal of vaccination. Four arms of experiment to vary the level of information the band signal gives.

\hfill

Sample Size:randomization in 120 clinics to affect 578 communities. 7400 - 2000 individuals depending on vaccination amounts.

\hfill

Estimated Effect \& Standard Errors(SE): 
The signaling at the 4th and 5th band treatment arms increased share of vaccination in communities.  Signal at the fifth band increased the timeliness of vaccination in communities that were treated.Share of children with 4 on time vaccinations for 5 vaccine signals 0.106 ∗(0.038).  Share of children with all 5 on time  vaccinations for 5 vaccine signals  0.137 (0.043) 

\hfill



\hfill

Any HTE: No

\textit{Subsidies, Information, and the Timing of Children’s Health Care in Mali
}


\hfill

Country: Sikoro,Mali

\hfill

Research Question:
 If subsidies supplemented with  information can curb under-use of health care services, without creating overuse of those same services.


\hfill

Type of Intervention: Randomly assigned our sample population to one of three treatment groups – healthworker visits, free care, or both combined – and a control group. 4 Arm study to piece out effects of information, subsides, info + subsides and a control.

\hfill

Sample Size:1544 children
\hfill

Estimated Effect \& Standard Errors(SE): 

Many results not sure what to hightailed.
Subsides increase demand and have no impact on overuse, point to reductions in under utilization of health services.
Info + subsides does similar outcomes as just subsides.
Info arm solely, seems to point to slight increase in under utilization people can better determine if sickness needs medical help.

\hfill


They estimate demand conditional on health status for diseases. This may be interesting.
\hfill

Any HTE: No


\textbf{JPAL Papers}


\textit{Subsidies, Information, and the Timing of Children’s Health Care in Mali
}


\hfill

Country: Sikoro,Mali

\hfill

Research Question:
 If subsidies supplemented with  information can curb under-use of health care services, without creating overuse of those same services.


\hfill

Type of Intervention: Randomly assigned our sample population to one of three treatment groups – healthworker visits, free care, or both combined – and a control group. 4 Arm study to piece out effects of information, subsides, info + subsides and a control.

\hfill

Sample Size:1544 children
\hfill

Estimated Effect \& Standard Errors(SE): 

Many results not sure what to hightailed.
Subsides increase demand and have no impact on overuse, point to reductions in under utilization of health services.
Info + subsides does similar outcomes as just subsides.
Info arm solely, seems to point to slight increase in under utilization people can better determine if sickness needs medical help.

\hfill


They estimate demand conditional on health status for diseases. This may be interesting.
\hfill

Any HTE: No

\textit{Mixed-Method Evaluation of a Passive mHealth Sexual Information Texting Service in Uganda
}


\hfill

Country: Uganda

\hfill

Research Question:
 Is merely passively improving access to  information sufficient to further specific social health goals? This case does just having access to information about sexual education increase safer sex practices.

\hfill

Type of Intervention: Created a  interactive text-messaging platform in Uganda for mobile telephones. This platform let people ask questions regarding sexual health by text. The treatment was to heavily adversities the use and services of the platform in treated villages.

\hfill

Sample Size: 2275
\hfill

Estimated Effect \& Standard Errors(SE): 

Non-promiscuity index -0.1096(0.042)
Ever had sex in past 12 months -0.0393(0.022)

Perceived relative non-riskiness index -0.1010(0.04)


\hfill
People did not practice safer sex and knowledge retention did not seem to be persistent. Passive knowledge sources did not succeed in making substantial change



\hfill

Any HTE: tested to see if by gender, but no difference between men and women.

\hfill

\textit{Mixed-Method Evaluation of a Passive mHealth Sexual Information Texting Service in Uganda
}


\hfill

Country: Uganda

\hfill

Research Question:
 Is merely passively improving access to  information sufficient to further specific social health goals? This case does just having access to information about sexual education increase safer sex practices.

\hfill

Type of Intervention: Created a  interactive text-messaging platform in Uganda for mobile telephones. This platform let people ask questions regarding sexual health by text. The treatment was to heavily adversities the use and services of the platform in treated villages.

\hfill

Sample Size: 2275
\hfill

Estimated Effect \& Standard Errors(SE): 

Non-promiscuity index -0.1096(0.042)
Ever had sex in past 12 months -0.0393(0.022)

Perceived relative non-riskiness index -0.1010(0.04)


\hfill
People did not practice safer sex and knowledge retention did not seem to be persistent. Passive knowledge sources did not succeed in making substantial change



\hfill

Any HTE: tested to see if by gender, but no difference between men and women.

\textit{Impact of a Daily SMS Medication Reminder System on Tuberculosis Treatment Outcomes: A Randomized Controlled Trial
}



\hfill

Country: Karachi, Pakistan

\hfill

Research Question:
Measure the impact of Zindagi SMS, a two-way SMS reminder system, on treatment success of people with drug-sensitive tuberculosis.

\hfill

Type of Intervention: A two-arm, parallel design, effectiveness randomized controlled trial in Karachi, Pakistan. Individual participants were randomized to either Zindagi SMS or the control group. Zindagi SMS sent daily SMS reminders to participants and asked them to respond through SMS or missed (unbilled) calls after taking their medication. Non-respondents were sent up to three reminders a day

\hfill

Sample Size:  2,207 participants, with 1,110 randomized to Zindagi SMS and 1,097 to the control group
\hfill

Estimated Effect \& Standard Errors(SE): 
No difference between treatment and control completion of drug treatment. Could not find regression SE.



\hfill
They state future research should couple the sms with financial incentives.


\hfill

Any HTE: tested to see if by gender, but no difference between men and women.


\textit{Impact of a Daily SMS Medication Reminder System on Tuberculosis Treatment Outcomes: A Randomized Controlled Trial
}



\hfill

Country: Karachi, Pakistan

\hfill

Research Question:
Measure the impact of Zindagi SMS, a two-way SMS reminder system, on treatment success of people with drug-sensitive tuberculosis.

\hfill

Type of Intervention: A two-arm, parallel design, effectiveness randomized controlled trial in Karachi, Pakistan. Individual participants were randomized to either Zindagi SMS or the control group. Zindagi SMS sent daily SMS reminders to participants and asked them to respond through SMS or missed (unbilled) calls after taking their medication. Non-respondents were sent up to three reminders a day

\hfill

Sample Size:  2,207 participants, with 1,110 randomized to Zindagi SMS and 1,097 to the control group
\hfill

Estimated Effect \& Standard Errors(SE): 
No difference between treatment and control completion of drug treatment. Could not find regression SE.



\hfill
They state future research should couple the sms with financial incentives.


\hfill

Any HTE: tested to see if by gender, but no difference between men and women.


\bibliography{dr_consulta.bib} 
\bibliographystyle{apalike}

\end{document}
